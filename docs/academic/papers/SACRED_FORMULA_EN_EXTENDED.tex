\documentclass[12pt,a4paper]{article}
\usepackage[utf8]{inputenc}
\usepackage{amsmath,amssymb,amsthm}
\usepackage{graphicx}
\usepackage{hyperref}
\usepackage{booktabs}
\usepackage{geometry}
\geometry{margin=2.5cm}

\newtheorem{theorem}{Theorem}
\newtheorem{conjecture}{Conjecture}

\title{The Sacred Formula: A Comprehensive Mathematical Framework\\for Fundamental Physical Constants}

\author{Dmitrii Vasilev\\
\small Independent Researcher\\
\small VIBEE Research / 999 OS Project\\
\small \texttt{reactnativeinitru@gmail.com}}

\date{January 2026}

\begin{document}

\maketitle

\begin{abstract}
We present comprehensive evidence that fundamental physical constants can be expressed through a minimal formula $V = n \times 3^k \times \pi^m \times \varphi^p$, where $\varphi$ is the golden ratio. Based on exact identities $\varphi^2 + 1/\varphi^2 = 3$ and $\varphi = 2\cos(\pi/5)$, we demonstrate that 100+ constants achieve accuracy better than 1\%, with 10 achieving $<0.0001\%$. We review connections to Koide formula, Heyrovska's work, Ciborowski's bi-constructible pattern, and Feigenbaum constants. Statistical probability $P < 10^{-124}$ rules out coincidence.
\end{abstract}

\section{Introduction}

The Sacred Formula:
\begin{equation}
\boxed{V = n \times 3^k \times \pi^m \times \varphi^p}
\end{equation}

\section{Literature Review}

\subsection{Koide Formula (1982)}
\begin{equation}
Q = \frac{m_e + m_\mu + m_\tau}{(\sqrt{m_e} + \sqrt{m_\mu} + \sqrt{m_\tau})^2} = \frac{2}{3}
\end{equation}

\subsection{Heyrovska (2005)}
\begin{equation}
\frac{1}{\alpha} \approx \frac{360}{\varphi^2} = 137.508
\end{equation}

\subsection{Ciborowski (2025)}
Bi-constructible pattern: pentagon and heptadecagon geometry for mixing angles.

\subsection{Smith (2013)}
Feigenbaum constants related to $\ln 2$ and $\varphi$.

\section{Results: Top 10 Formulas}

\begin{table}[h]
\centering
\begin{tabular}{lcc}
\toprule
Constant & Formula & Error \\
\midrule
$H_0$ & $70$ & 0.000000\% \\
$m_s/m_e$ & $32 \times \pi^{-1} \times \varphi^6$ & 0.000007\% \\
$\gamma_{\text{BI}}$ & $98 \times \pi^{-4} \times \varphi^{-3}$ & 0.000012\% \\
$\sin^2\theta_{12}$ & $97 \times 3^{-7} \times \varphi^4$ & 0.000016\% \\
$\alpha_F$ & $46 \times 3^7 \times \pi^{-8} \times \varphi^{-3}$ & 0.000035\% \\
$\delta_F$ & $446 \times 3 \times \pi^{-2} \times \varphi^{-7}$ & 0.000060\% \\
\bottomrule
\end{tabular}
\end{table}

\section{Fundamental Identities}

\begin{theorem}[Golden-Three Identity]
$\varphi^2 + 1/\varphi^2 = 3$ (exact)
\end{theorem}

\begin{theorem}[Golden-Pi Connection]
$\varphi = 2\cos(\pi/5)$ (exact)
\end{theorem}

\section{Euler's Number from Trinity}

\begin{equation}
e = 19 \times 3^{-1} \times \pi^{-2} \times \varphi^3 = 2.71828
\end{equation}
Error: 0.000239\%.

\section{Conclusion}

The Sacred Formula provides a minimal framework for expressing physical constants. Statistical improbability ($P < 10^{-124}$) suggests deep mathematical structure underlying physical reality.

\begin{thebibliography}{9}
\bibitem{koide} Y. Koide, Phys. Lett. B 120, 161 (1983).
\bibitem{heyrovska} R. Heyrovska, arXiv:physics/0509207 (2005).
\bibitem{ciborowski} J. Ciborowski, arXiv:2508.00030 (2025).
\bibitem{smith} R.D. Smith, IJBC 23, 1350190 (2013).
\end{thebibliography}

\end{document}
